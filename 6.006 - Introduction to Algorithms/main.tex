\documentclass[11pt]{article}
\usepackage{lindrew}
\usepackage{listings}

\definecolor{backcolor}{rgb}{0.95,0.95,0.92}
\definecolor{codegreen}{rgb}{0,0.6,0}

\lstdefinestyle{cptbreeza}{
  keywordstyle=\underbar,
  commentstyle=\color{codegreen},
  basicstyle=\linespread{0.99}\ttfamily,
  captionpos=b,
  numbers=left,
  numbersep=10pt,
  frame=single,
}
\lstset{style=cptbreeza}

\newcommand{\Prob}[2]{\textbf{Input:} {#1} \\ \textbf{Output:} {#2}}

\newcommand{\Arr}[1]{\langle {#1} \rangle}

\declaretheoremstyle[
  headpunct=:,
  bodyfont=\normalfont
]{loopinvariant}

\declaretheoremstyle[
  headpunct=:,
  bodyfont=\normalfont
]{runtime}

\declaretheorem[style=loopinvariant, name=Loop Invariant, numbered=no]{inv*}
\declaretheorem[style=runtime, name=Run Time, numbered=no]{rtime*}

\title{6.006 - Introduction to Algorithms}
\author{CptBreeza}
\date{June 2024}

\begin{document}

\maketitle

\section{Insertion Sort}

\subsection{The Problem}

\Prob
{A sequence of $n$ numbers $\Arr{a_1,a_2,\dots,a_n}$.}
{A permutation $\Arr{a'_1,a'_2,\dots,a'_n}$ of the input sequence such that
$a'_1 \leq a'_2 \leq \dots \leq a'_n$.}

\subsection{Procedures}

\begin{lstlisting}[language=Python]
def insertion_sort(a, n):
    for i in range(1, n):
        key = a[i]
        j = i - 1
        while j > -1 and a[j] > key:
            a[j + 1] = a[j]
            j = j - 1
        a[j + 1] = key
\end{lstlisting}

\begin{inv*}
At the start of each iteration of the \textbf{for} loop of lines $2-8$, the subarray
$A[0:i-1]$ consists of the elements originally in $A[0:i-1]$, but in sorted order.
\end{inv*}

\begin{rtime*}
$\Theta(n^2)$.
\end{rtime*}

\subsection{Advantages and Disadvantages}
\textbf{Advantages:}
\begin{enumerate}
\item In place sorting (take no extra memories).
\end{enumerate}
\textbf{Disadvantages:}
\begin{enumerate}
\item Non-optimal running time.
\end{enumerate}


\section{Merge Sort}

\subsection{The Problem}

\Prob
{A sequence of $n$ numbers $\Arr{a_1,a_2,\dots,a_n}$.}
{A permutation $\Arr{a'_1,a'_2,\dots,a'_n}$ of the input sequence such that
$a'_1 \leq a'_2 \leq \dots \leq a'_n$.}

\subsection{Procedures}

\begin{lstlisting}[language=Python]
def merge(a, p, q, r):
    n_l = q - p + 1
    n_r = r - q
    a_l = a[p:p + n_l]
    a_r = a[p + n_l:p + n_l + n_r]
    i = 0
    j = 0
    k = p

    while i < n_l and j < n_r:
        if a_l[i] <= a_r[j]:
            a[k] = a_l[i]
            i = i + 1
        else:
            a[k] = a_r[j]
            j = j + 1
        k = k + 1

    while i < n_l:
        a[k] = a_l[i]
        i = i + 1
        k = k + 1
    while j < n_r:
        a[k] = a_r[j]
        j = j + 1
        k = k + 1
\end{lstlisting}

\begin{rtime*}
$\Theta(n)$.
\end{rtime*}

\begin{lstlisting}[language=Python]
def merge_sort(a, p, r):
    if p >= r:
        return

    q = math.floor((p + r) / 2)
    merge_sort(a, p, q)
    merge_sort(a, q + 1, r)
    merge(a, p, q, r)
\end{lstlisting}

\begin{rtime*}
$\Theta(n \cdot log(n))$.
\end{rtime*}

\subsection{Advantages and Disadvantages}

\textbf{Advantages:}
\begin{enumerate}
\item Better running time.
\end{enumerate}
\textbf{Disadvantages}
\begin{enumerate}
\item Need extra memories while sorting.
\end{enumerate}


\section{Heap and Heapsort}

\subsection{The Problem}

Sorting problem.

\subsection{Data Structures}

\subsubsection{Binary Heap}

The \emph{(binary) heap} data structure is an array that we can view as a nearly
complete binary tree. The tree is completely filled on all levels except possibly
the lowest, which is filled from left up to a point.

The root of the tree is $A[1]$, and given the index $i$ of a node, we can compute
the indices of its parent, left child, right child:
\begin{lstlisting}[language=Python]
def parent(i):
    return math.floor(n / 2)

def left(i):
    return 2 * i

def right(i):
    return 2 * i + 1
\end{lstlisting}

\begin{proof}
First we prove $left(i)$ and $right(i)$. We prove it by induction.

\textbf{Base case:} $i=1$ implies $left(1)=2$ and $right(1)=3$, which is correct.

\textbf{Inductive case}: Suppose $left(i)=2i$ and $right(i)=2i+1$, we need to prove
that $left(i+1)=2(i+1)$ and $right(i+1)=2(i+1)+1$.

Since the children of the $i$th node are at $2i$ and $2i+1$, the next two elements
after are $2i+2$ and $2i+3$, and by the definition of binary heap they are exactly
the left child and the right child of the $i+1$th node. Rewrite the indices of them
as $2i+2=2(i+1)$ and $2i+3=2(i+1)+1$, this is exactly what we want to prove.

For a non-root node $i$, it is either a left child or a right child of its parent
node. If it is the left child, its parent node is at $\frac{i}{2}$. If it is the
right child, its parent node is at $\frac{i-1}{2}$. In each case, its parent is at
$\lfloor\frac{i}{2}\rfloor$.
\end{proof}

The \emph{height of a node} in a heap is the number of edges on the longest simple
downward path from the node to a leaf, and the \emph{height of the heap} is the
height of its root.

An $n$-element heap has height $\lfloor \lg n \rfloor$.

\begin{proof}
As a binary heap is a nearly complete binary tree, for a heap of height $h$, it has
at most $2^0+2^1+\dots+2^h=2^{h+1}-1$ elements when it is complete, and at least
$2^0+2^1+\dots+2^{h-1}+1=2^h$ elements when there is only one element in the last
level.

For an $n$-element, $2^h \leq n \leq 2^{h+1}-1$ implies
$h \leq \lg n \leq \lg(2^{h+1}-1) < \lg 2^{h+1} = h+1$. Since $h \leq \lg n < h+1$,
we know that $h=\lfloor \lg n \rfloor$.
\end{proof}

\subsubsection{Max-Heap and Min-Heap}

There are two kinds of binary heaps, max-heap and min-heap, each of which satisfies
a \emph{heap property}. In a max-heap, the \emph{max-heap property} is that for
every node $i$ other than the root, $A[parent(i)] \geq A[i]$.

This implies that the largest element in a max-heap is stored at the root, and the
subtree rooted at a node contains values no larger than that contained at the root
itself.

\subsection{Procedures}

\subsubsection{Maintaining the Max-Heap Property}

Assume subtrees rooted at $left(i)$ and $right(i)$ are max-heaps, the following
procedure \textbf{max-heapify} makes the subtree rooted at $i$ a max-heap.

\begin{lstlisting}[language=Python]
def max_heapify(a, i, n):
    left = 2 * i + 1
    right = 2 * i + 2

    j = left if left <= n and a[left] > a[i] else i
    j = right if right <= n and a[right] > a[j] else j

    if i != j:
        a[i], a[j] = a[j], a[i]
        max_heapify(a, j, n)
\end{lstlisting}

\begin{rtime*}
Suppose the subtree rooted at $i$ has length $n$ and height $h$, then
$T(n) = \Theta(\lg n) = \Theta(h)$.
\end{rtime*}

\subsubsection{Building a Max-Heap}

\begin{lstlisting}[language=Python]
def build_max_heap(a, n):
    m = math.floor((n + 1) / 2) - 1

    for i in range(m + 1):
        max_heapify(a, m - i, n)
\end{lstlisting}

\begin{inv*}
At the start of each iteration of the \textbf{for} loop of lines $4-5$, every subtree
rooted at $i > \lfloor\frac{n+1}{2}\rfloor - 1$ is a max-heap.
\end{inv*}

\begin{rtime*}
$\Theta(n)$.
\end{rtime*}

\subsubsection{Sorting with Max-Heaps}

\begin{lstlisting}[language=Python]
def heapsort(a, n):
    build_max_heap(a, n)

    for i in range(n, 0, -1):
        a[0], a[i] = a[i], a[0]
        n = n - 1
        max_heapify(a, 0, n)
\end{lstlisting}

\begin{rtime*}
$\Theta(n\lg n)$.
\end{rtime*}

\subsection{Advantages and Disadvantages}

\textbf{Advantages:}
\begin{enumerate}
\item Better running time.
\item In place sorting.
\end{enumerate}


\section{Binary Search Trees}

\subsection{The Problem}

Sorting problem.

\subsection{Data Structure}

A binary search tree is a binary tree with the property: for a node $x$, if $y$ is a
node in the left subtree of $x$, then $y.key \leq x.key$; if $y$ is a node in the
right subtree of $x$, then $y.key \geq x.key$.

\subsection{Procedures}

\subsubsection{Print Keys in Sorted Order}

\begin{lstlisting}[language=Python]
def inorder_tree_walk(node):
    if node is not None:
        inorder_tree_walk(node.left)
        print(node.key)
        inorder_tree_walk(node.right)
\end{lstlisting}

\begin{rtime*}
For a tree with $n$ nodes, it takes $\Theta(n)$.
\end{rtime*}

\subsubsection{Searching}

\begin{lstlisting}[language=Python]
def tree_search(node, k):
    if node is None or k == node.key:
        return node

    if k < node.key:
        return tree_search(node.left, k)
    else:
        return tree_search(node.right, k)
\end{lstlisting}

\begin{rtime*}
For a tree of height $h$, it takes $\Theta(h)$.
\end{rtime*}

\subsubsection{Maximum and Minimum}

\begin{lstlisting}[language=Python]
def tree_minimum(node):
    while node.left is not None:
        node = node.left
    return node


def tree_maximum(node):
    while node.right is not None:
        node = node.right
    return node
\end{lstlisting}

\begin{rtime*}
$\Theta(h)$.
\end{rtime*}

\subsubsection{Successor and Predecessor}

\begin{lstlisting}[language=Python]
def tree_successor(node):
    if node.right is not None:
        return tree_minimum(node.right)

    while node.parent is not None and node != node.parent.left:
        node = node.parent

    return node.parent

def tree_predecessor(node):
    if node.left is not None:
        return tree_maximum(node.left)

    while node.parent is not None and node != node.parent.right:
        node = node.parent

    return node.parent
\end{lstlisting}

\begin{rtime*}
$\Theta(h)$.
\end{rtime*}

\subsubsection{Insertion}

This procedure searches for a appropriate slot all the way down to leaves.

\begin{lstlisting}[language=Python]
def tree_insert(root, node):
    x = root
    y = None
    while x is not None:
        y = x
        x = x.left if node.key < x.key else x.right
    if node.key < y.key:
        y.left = node
    else:
        y.right = node
\end{lstlisting}

\begin{rtime*}
$\Theta(h)$.
\end{rtime*}

\subsubsection{Deletion}

\begin{lstlisting}[language=Python]
def tree_delete(root, node):
    attr = "left" if node == node.parent.left else "right"

    match (node.left, node.right):
        case (None, r):
            node.parent.__setattr__(attr, r)

        case (l, None):
            node.parent.__setattr__(attr, l)

        case (l, r):
            s = tree_successor(node)

            if s == node.right:
                s.left = node.left
                node.parent.__setattr__(attr, s)
            else:
                s_attr = "left" if s == s.parent.left else "right"
                s_r = s.right
                s_p = s.parent

                # replace `node` with `s`
                s.left = node.left
                s.right = node.right
                node.parent.__setattr__(attr, s)

                # replace the original `s` with its right child
                s_p.__setattr__(s_attr, s_r)
\end{lstlisting}

\begin{rtime*}
$\Theta(h)$.
\end{rtime*}

\end{document}
