\documentclass[11pt]{article}
\usepackage{lindrew}
\usepackage{proof}

\newcommand{\Zn}{\mathbb{Z}_n}
\newcommand{\Zns}{\mathbb{Z}^\ast_n}
\newcommand{\Zs}{\mathbb{Z}^\ast}
\newcommand{\Dotn}{\cdot_n}
\newcommand{\Inzns}{\;\;(\Zn)}

\title{6.042J - Mathematics for Computer Science}
\author{CptBreeza}
\date{June 2024}

\begin{document}

\maketitle

\section{Proof Techniques}

\subsection{Proof by Contradiction}

Normally if we want to prove a proposition $P$, we assume its negation $\neg P$ and derive
a contradiction, showing that it is only possible that $P$ is true.

The following inference justifies this methodology:

\[
\infer{P}{\neg P \Rightarrow \text{contraidtion}}
\]

\subsubsection{Examples}

Prove that $\sqrt{2}$ is not a rational number.

\begin{proof}
We prove it by contradiction. Suppose that $\sqrt{2}$ \emph{is} a rational number. Every rational number can be written as a fraction $\frac{s}{t}$ where $s,t \in \mathbb{Z}$. We can assume that $s$ has no common divisor with $t$, for if it does, we can divide both $s$
and $t$ with the common divisor and get a pair of other integers $s'$ and $t'$, then we
will use them.

If $s$ has no common divisor with $t$, $s$ and $t$ cannot be both even numbers. A even
number can be divided by $2$. If they are both even numbers, they share a common divisor
$2$. But the following reasoning show that both $s$ and $t$ must be even numbers:
\begin{align*}
\frac{s}{t} &= \sqrt{2} \\
s &= \sqrt{2}t \\
s^2 &= 2t^2
\end{align*}

The last equation shows that $s^2$ is an even number, since it is a multiple of $2$. If
$s^2$ is an even number, $s$ must also be an even number. Say $s = 2s'$. Now we have
\begin{align*}
(2s')^2 &= 2t^2 \\
4s'^2 &= 2t^2 \\
2s'^2 &= t^2
\end{align*}

We see that this time $t^2$ is a multiple of $2$. By the same argument, $t$ must be an even
number. We derived that both $s$ and $t$ must be both even numbers, which contradicts with
the previous assumption that $s$ and $t$ share no common divisor. So $\sqrt{2}$ is not a
rational number.

\end{proof}

\subsection{The Well-Ordering Principle}

There is a basic fact about the natural numbers:

\begin{theorem}[The Well-Ordering Principle]
Every non-empty subset of natural numbers $S \subseteq \mathbb{N}$ contains a least element.
\end{theorem}

This fact is useful when we want to prove a proposition about natural numbers and it is
usually used together with proof by contradiction.

\subsubsection{Examples}

Prove that any positive numbers $s$ and $t$ can be written in \emph{lowest terms}, that is,
in the form $\frac{s'}{t'}$ where $s'$ and $t'$ are positive integers with no common
divisors.

\begin{proof}
We prove it by contradiction. Suppose that there \emph{are} positive numbers $s$ and $t$
such that $\frac{s}{t}$ cannot be written in lowest terms. Let $C$ be the set of positive
integers that are numerators of such fractions. $s \in C$ so $C$ is not empty.

By Well Ordering Principle, $C$ has a smallest element $s_0$. By the definition of $C$,
there is a fraction $\frac{s_0}{t_0}$ that cannot be written in lowest terms. What it means
is that $m_0$ and $n_0$ has a common divisor $p > 1$. Now we have
\[
  \frac{m_0/p}{n_0/p} = \frac{m_0}{n_0}
\]
That is to say that $\frac{m_0/p}{n_0/p}$ cannot be written in lowest terms either. By the
definition of $C$, $m_0/p$ is in $C$. But $m_0/p < m_0$, which contradicts the fact that
$m_0$ is the smallest element of $C$.

\end{proof}

\subsection{Proof by Induction}

Another way to prove propositions about natural numbers is by induction.

\begin{theorem}[The Induction Principle]
Let $P$ be a predicate on natural numbers. If
\begin{itemize}
\item $P(0)$ is true, and
\item if $P(n)$ is true, $P(n+1)$ is true for all natural number $n$
\end{itemize}
Then $P(m)$ is true for all natural numbers.
\end{theorem}

\subsubsection{Examples}

Prove that for any natural number $n$, $\Sigma_{i=1}^n i = \frac{n(n+1)}{2}$.

\begin{proof}
We prove it by induction on $n$.

\textbf{Base case:} $n = 0$. For $n = 0$, there are no terms in the summation on the left
hand side, so it is equal to $0$. The right hand side also gives a $0$. So the formula
holds.

\textbf{Inductive case:} We \emph{assume} that $\Sigma_{i=1}^n i = \frac{n(n+1)}{2}$ and we
want to \emph{prove} that $\Sigma_{i=1}^{n+1} i = \frac{(n+1)(n+2)}{2}$.

We expand the left hand side of the equation that we want to prove and apply the induction
hypothesis
\begin{align*}
\Sigma_{i=1}^{n+1} i &= (\Sigma_{i=1}^n i) + (n+1) \\
&= \frac{n(n+1)}{2} + (n+1) \\
&= \frac{n(n+1)+2(n+1)}{2} \\
&= \frac{(n+1)(n+2)}{2}
\end{align*}
and finally established that $\Sigma_{i=1}^{n+1} i = \frac{(n+1)(n+2)}{2}$.

\end{proof}

\subsection{A Variant of Proof by Induction: Strong Induction}

In the inductive case of a proof by induction, we assume $P(n)$ and need to prove $P(n+1)$.
In fact, we can use a much stronger assumption.

\begin{theorem}[Principle of Strong Induction]
Let $P$ be a predicate on natural numbers. If
\begin{itemize}
\item $P(0)$ is true, and
\item if $P(0), P(1), \dots, P(n)$ is true, $P(n+1)$ is true for all natural number $n$
\end{itemize}
then $P(m)$ is true for all natural numbers.
\end{theorem}

\section{Number Theory}

In this section, the default set of variables is $\mathbb{Z}$.

\subsection{Divisibility}

\begin{definition}
$a$ \emph{divides} $b$ (notation $a|b$) iff there is an integer $k$ such that $ak=b$.
\end{definition}

\subsubsection{Facts}

\begin{enumerate}
\item If $a|b$ and $b|c$, then $a|c$.
\begin{proof}
$a|b$ so $b=ka$. $b|c$ so $c=k'b$. Combine them to get $c=k'ka$.
\end{proof}

\item If $a|b$ and $a|c$, then $a|sb+tc$ for all $s$ and $t$.
\begin{proof}
$a|b$ and $a|c$ so $sb+tc=ska+tk'a=(sk+tk')a$.
\end{proof}

\item For all $c \neq 0$, $a|b$ if and only if $ca|cb$.
\begin{proof}
\textbf{The only-if part:} $a|b$ so $b=ka$. $b=ka$ so $cb=cka=kca$, which is $ca|cb$.
\textbf{The if part:} $ca|cb$ so $cb=kca$. $cb=kca$ and $c \neq 0$ so $b=ka$, which is
$a|b$.
\end{proof}

\end{enumerate}

\begin{theorem}[Division Theorem]
Let $n$ and $d$ be integers such that $d > 0$. Then there exists a unique pair of integers
$q$ and $r$, such that $n = qd+r$ and $0 \leq r < d$.
\end{theorem}

\begin{proof}
We split the proof into two parts. One part on existence and one part on uniqueness.

First, we prove that there exists such a pair $q$ and $r$. We prove it by strong induction
on $n$.

\textbf{Base Case:} If $0 \leq n < d$, $q = 0$ and $r = n$.

\textbf{Inductive Case:} We assume the proposition holds for all integers $[0,n-1]$ and
$n \geq d$. We need to prove that the proposition holds for $n$. Since $0 \leq n-d < n$,
by the induction hypothesis, $n-d=qd+r$ and $0 \leq r < d$, we get $n=(q+1)d+r$.

Then we prove that if there exists two pairs $q,r$ and $q',r'$ such that $n=qd+r=q'd+r'$
and $0 \leq r,r' < d$, it must be that $q=q'$ and $r=r'$. We prove it by contradiction.

We assume that $0 \leq r \leq r' < d$. We have $r'-r=kd$ where $k = q'-q$. If $q'=q$ then
$r'$ must be equal to $r$. If $q' \neq q$, since $r' \geq r$, it must be that $k > 1$.
But $r' > r'-r=kd$ implies that $r' > kd > d$, which contradicts our assumption that
$r' < d$.

\end{proof}

\subsection{Greatest Common Divisor}

\begin{lemma} \label{lemma:6}
For $b \neq 0$, $gcd(a,b) = gcd(b,rem(a,b))$.
\end{lemma}

\begin{proof}
We rewrite $a$ as $a = qb+r$ where $0 \leq r < b$. Since $a$ is a linear combination of $b$
and $r$, any common divisor of $b$ and $r$ is a also a divisor of $a$. This also means that
any common divisor of $b$ and $r$ is also a common divisor of $a$ and $b$. Similarly, since
$r = -qb+a$, any common divisor of $a$ and $b$ is also a divisor of $r$ and $b$.
\end{proof}

\begin{theorem} \label{theorem:7}
Th greatest common divisor of $a$ and $b$ is a linear combination of $a$ and $b$.
\end{theorem}

\begin{proof}
Any common divisor would divide any linear combination of $a$ and $b$. So is the greatest
commont divisor.
\end{proof}

\subsection{The Fundamental Theorem of Arithmetic}

\begin{lemma}
If $p$ is a prime and $p|ab$, then $p|a$ or $p|b$.
\end{lemma}

\begin{proof}
If $p$ divides $a$, we are done. Then the other case is that $p$ does not divide $a$. That
is to say, $gcd(a,p)=1$ and we have $sa+tp=1$. We multiply both sides by $b$ and get
$sab+tpb=b$. $p|sab$ since $p|ab$ and $p|tbp$, which implies $p|b$.
\end{proof}

\begin{lemma} \label{lemma:9}
Every positive integer can be expressed as a product of primes.
\end{lemma}

\begin{proof}
We prove it by contradiction. Suppose for contrary there is a positive number $s$ that can
not be expressed as a product of primes. Let $C$ be the set of positive numbers that cannot
be expressed as a product of primes. $s \in C$ so $C$ is not empty.

By Well Ordering Principle, $C$ has a smallest element $s_0$. By the definition of $C$,
$s_0$ cannot be expressed as a product of primes. Then $s_0$ cannot be a prime number, since
a prime is a product of primes. So $s_0$ must be a product of two integers $1 < a,b < s_0$.
And $a$ can be written as a product of primes $p_1p_2 \dots p_i$ and $b$ can be written as a
product of primes $q_1q_2 \dots q_j$. $s_0 = ab = p_1 \dots p_iq_1 \dots q_j$. That is,
$s_0$ can be expressed as a product of primes, which contradicts the claim $s_0 \in C$.
\end{proof}

\begin{theorem}[Fundamental Theorem of Arithmetic]
Every positive number is a product of a unique weakly decreasing sequence of primes.
\end{theorem}

\begin{proof}
We split the proof into two parts. One part for existence and one part for uniqueness.

First we prove that there is such a sequence. By lemma \ref{lemma:9} every positive can be
expressed as a product of primes. Then we can write those primes divisors in a weakly
decreasing sequence.

Then we prove the sequence is unique. We prove it by contradiction. Assume that there are numbers that are products of a distinct weakly decreasing sequence of primes. Let $n$ be
the smallest of them and
\begin{align*}
n &= p_1p_2 \dots p_i \\
&= q_1q_2 \dots q_j
\end{align*}

If $p_1 = q_1$, $n/q_1 = p_2 \dots p_i = q_2 \dots q_j$ is still products of distinct
sequences. But $n/q_1 < n$, which is not possible by the Well Ordering Principle. So it must
be $p_1 \neq q_1$.

If $p_1 \neq q_1$, we assume that $p_1 < q_1$. Since the sequence is weakly decreasing, all
$p_x < q_1$. The fact that $q_1|p_1p_2 \dots p_i$ implies that $q_1|p_x$, but $q_1 > p_x$,
which is not possible.

\end{proof}

\subsection{Modular Arithmetic}

\begin{definition}
$a$ is \emph{congruent to} $b$ \emph{modulo} $n$ iff $n|(a-b)$. This is written as
$$
a \equiv b \pmod{n}
$$
\end{definition}

\begin{lemma} \label{lemma:12}
$a \equiv b \pmod{n}$ iff $rem(a, n) = rem(b,n)$.
\end{lemma}

\begin{proof}
We split the proof into two parts. One part for the if direction and one part for the
only-if direction.

\textbf{The proof of the if part}: We can write $a = pn + r$ and $b = p'n + r$ where
$0 \leq r < n$. Subtract these two equations and we get $a - b = (p - p')n$ which means
that $n|(a-b)$.

\textbf{The proof of the only-if part}: Suppose $a=pn+r$ and $b=p'n+r'$ where
$0 \leq r,r' < n$. Subtract these two equations and we get $a-b=(p-p')n+(r-r')$. Since
$n|(a-b)$ and $p|(p-p')n$, $n|((a-b)-(p-p')n)$, which is $n|(r-r')$. We known that
$0 \leq r,r' < n$, so it is only possible that $r-r'=0$ and $r=r'$.

\end{proof}

\begin{lemma}
Some properties of $\equiv \pmod{n}$:
\begin{align*}
a &\equiv a \pmod{n} \tag{reflexivity} \\
a \equiv b \text{ IFF } b &\equiv a \pmod{n} \tag{symmetry} \\
\text{(}a \equiv b \text{ AND } b \equiv c \text{)} \text{ IMPLIES } a &\equiv c \pmod{n}
\tag{transitivity}
\end{align*}
\end{lemma}

\begin{proof}
These properties can easily be proved using lemma \ref{lemma:12}. \\
\textbf{reflexivity}: $rem(a,n) = rem(a,n)$. \\
\textbf{symmetry}: $rem(a,n)=rem(b,n) \Leftrightarrow rem(b,n)=rem(a,n)$. \\
\textbf{transitivity}: $rem(a,n)=rem(b,n) \wedge rem(b,n)=rem(c,n) \Rightarrow rem(a,n)=
rem(c,n)$.
\end{proof}

\begin{lemma} \label{lemma:14}
$a \equiv rem(a,n) \pmod{n}$.
\end{lemma}

\begin{proof}
Let $a = pn + r$ where $0 \leq r < n$, then $a-rem(a,n)=pn+r-r=pn$, which means
$p|(a-rem(a,n))$.
\end{proof}

\begin{lemma} \label{lemma:15}
If $a \equiv b \pmod{n}$ and $c \equiv d \pmod{n}$, then
\begin{enumerate}
\item $a + c \equiv b + d \pmod{n}$,
\item $ac \equiv bd \pmod{n}$.
\end{enumerate}
\end{lemma}

\begin{proof}
By subtraction and regrouping we get $(a+c)-(b+d)=(a-b)+(c-d)$. Since $p|(a-b)$ and
$p|(c-d)$, $p|((a-b)+(c-d))$, which can be rewritten as $p|((a+c)-(b+d))$.

From $p|(b-a)$ we have $p|(b-a)c$. From $p|(c-d)$ we have $p|(d-c)b$. Then we have
$p|( (b-a)c + (d-c)b )$, which is $p|(bd-ac)$.
\end{proof}

\subsection{Remainder Arithmetic}

\begin{definition}
\begin{align*}
i +_n j &::= rem(i+j, n) \\
i \cdot_n j &::= rem(ij, n)
\end{align*}
\end{definition}

\begin{lemma} \label{lemma:17}
$rem(i+j,n) = rem(i,n) +_n rem(j,n)$.
\end{lemma}


\begin{proof}
Expand $+_n$ on the right hand side shows what we want to prove is:
\[
  rem(i+j,n) = rem(rem(i,n)+rem(j,n),n)
\]

By lemma \ref{lemma:14}, $i \equiv rem(i,n) \pmod{n}$ and $j \equiv rem(j,n) \pmod{n}$. And
by lemma \ref{lemma:15}, $i+j \equiv rem(i,n)+rem(j,n) \pmod{n}$. By lemma \ref{lemma:12},
$i+j \equiv rem(i,n) + rem(j,n) \pmod{n}$ implies that
$rem(i+j,n) = rem((rem(i,n) + rem(j,n), n)$, which is exactly what we want to prove.
\end{proof}

\begin{lemma} \label{lemma:18}
$rem(ij,n) = rem(i,n) \cdot_n rem(j,n)$.
\end{lemma}

\begin{proof}
Expand $\cdot_n$ on the right hand side shows what we want to prove is:
\[
  rem(ij,n) = rem(rem(i,n) \cdot rem(j,n),n)
\]

By lemma \ref{lemma:14} and lemma \ref{lemma:15}, we have
$ij = rem(i,n) \cdot rem(j,n) \pmod{n}$. By lemma \ref{lemma:12},
$ij = rem(i,n) \cdot rem(j,n) \pmod{n}$ implies that
$rem(ij,n) = rem(rem(i,n) \cdot rem(j,n), n)$, which is exactly what we want to prove.
\end{proof}

The set of integers in the range $[0, n)$ together with the operations $+_n$ and $\cdot_n$
is referred to as $\mathbb{Z}_n$, the \emph{ring of integers modulo} $n$. The $+_n$ is
associative and commutative. The $\cdot_n$ is associative and commutative. $+_n$ has an
identity and an inverse. $\cdot_n$ has an identity. And $\cdot_n$ distributes oveer $+_n$.

\subsection{Multiplicative Inverses and Cancelling}

\begin{definition}
Integers that have no prime factor in common are called \emph{relatively prime}.
\end{definition}

\begin{lemma}
If $k \in [0, n)$ is relatively prime to $n$, then $k$ has an inverse in $\mathbb{Z}_n$.
\end{lemma}

\begin{proof}
If $k$ is relatively prime to $n$, by definition $gcd(n,k) = 1$. By theorem
\ref{theorem:7}, there is a linear combination $sn+tk=1$. By lemma \ref{lemma:17} and lemma
\ref{lemma:18} we get
\begin{align*}
  (rem(s,n) \cdot rem(n,n)) + (rem(t,n) \cdot rem(k,n)) &= 1 \;\;(\mathbb{Z}_n) \\
  rem(t,n) \cdot k &= 1 \;\;(\mathbb{Z}_n)
\end{align*}
So we have $rem(t,n)$ as a multiplicative inverse of $k$.
\end{proof}

\begin{lemma}
If $i$ and $j$ are both inverses of $k$ in $\mathbb{Z}_n$, then $i = j$.
\end{lemma}

\begin{proof}
$i = i \cdot 1 = i \cdot (k \cdot j) = (i \cdot k) \cdot j = 1 \cdot j = j
\;\;(\mathbb{Z}_n)$.
\end{proof}

\begin{definition}
A number $k$ is \emph{cancellable} in $\mathbb{Z}_n$ iff
\[
  k \cdot b = k \cdot a \Rightarrow b = a \;\;(\mathbb{Z}_n)
\]
for all $a, b \in [0, n)$.
\end{definition}

\begin{lemma}
If $k$ has an inverse in $\mathbb{Z}_n$, then it is cancellable.
\end{lemma}

\begin{proof}
For any $a, b \in \mathbb{Z}_n$, if $k \cdot a = k \cdot b \;\;(\mathbb{Z}_n)$, we multiply
$k^{-1}$ on both sides
\begin{align*}
k^{-1} \cdot k \cdot a &= k^{-1} \cdot k \cdot b \\
a &= b
\end{align*}
\end{proof}

\subsection{Euler's Theorem}

\begin{definition}
For all $n > 0$, define
\[
  \Phi(n) ::= \text{the number of integers in } [0, n) \text{ that is relatively prime to }
  n
\]
\end{definition}

\begin{definition}
Let $\Zns$ be the set of integers in $(0,n)$, that are relatively prime to $n$.
\end{definition}

By the definition we have $\Phi(n) = |\Zns|$.

\begin{lemma}
If $j,k \in \Zns$, then $j \cdot_n k \in \Zns$.
\end{lemma}

\begin{proof}
Since $j$ shares no common divisor with $n$ and $k$ shares no common divisor with $n$, $jk$
also shares no common divisor with $n$. That is to say, $gcd(jk,n)=1$. By Lemma
\ref{lemma:6} we have $gcd(jk,n) = gcd(n,rem(jk,n))=1$, which implies $rem(jk,n)$ is
relatively prime to $n$. Since $rem(jk,n) = j \cdot_n k$ and it is relatively prime to n,
$j \cdot_n k \in \Zns$.
\end{proof}

\begin{definition}
For any element $k$ and subset $S$ of $\Zns$, let $kS = \{k \cdot_n s \mid s \in S\}$.
\end{definition}

\begin{lemma} \label{lemma:26}
If $k \in \Zns$ and $s \subseteq \Zn$, then $|kS| = |S|$.
\end{lemma}

\begin{proof}
For any pair elements $k \Dotn a$ and $k \Dotn b$ in $S$, since $k$ is cancellable in $\Zn$
\[
  k \Dotn a = k \Dotn b \Rightarrow a = b
\]
That is to say multiplication by $k$ sends distinct elements in $S$ to distinct elements in
$kS$. So $kS$ and $S$ are the same size.
\end{proof}

\begin{lemma} \label{lemma:28}
If $k \in \Zns$, $k\Zns = \Zns$.
\end{lemma}

\begin{proof}
We split the proof into two parts. First we show that $k\Zns$ and $\Zns$ are the same size.
Then we show that $k\Zns$ is a subset of $\Zns$. These two facts imply that $k\Zns = \Zns$.

Since $\Zns$ is a subset of $\Zn$ and $k \in \Zn$, by lemma \ref{lemma:28}, $k\Zns$ and
$\Zns$ are the same size.

Every element of $k\Zns$, say $k \Dotn k_i$, is in $\Zns$ by lemma \ref{lemma:26}. That is
to say that every element of $k\Zns$ is also an element of $\Zns$. So $k\Zns$ is a subset of
$\Zns$.
\end{proof}

\begin{theorem}[Euler's Theorem]
For all $k \in \Zns$, $k^{\Phi(n)} \equiv 1 \pmod{n}$.
\end{theorem}

\begin{proof}
Let
\[
  P = k_1k_2 \dots k_{\Phi(n)} \Inzns
\]
and
\[
  Q = (k \cdot k_1)(k \cdot k_2) \dots (k \cdot k_{\Phi(n)}) \Inzns
\]

$P$ is the product of all elements of $\Zns$ and $Q$ is the product of all elements of
$k\Zns$ and $Q = k^{\Phi(n)}P \Inzns$. Since $k\Zns = \Zns$, $Q$ is also the products of all
elements of $\Zns$. So we have
\[
  P = Q = k^{\Phi(n)}P \Inzns
\]

Furthermore by lemma \ref{lemma:26} $P \in \Zns$, we can cancel it, giving
\[
  1 = k^{\Phi(n)} \Inzns
\]
\end{proof}

Euler's Theorem offers a way to compute inverses modulo $n$: if $k$ is relatively prime to
$n$, then $k^{\Phi(n)}$ is a $\Zn$-inverse of $k$.

\subsubsection{Computing Euler's $\Phi$ Function}

\begin{lemma}
$\Phi(pq) = (p-1)(q-1)$ for primes $p \neq q$.
\end{lemma}

\begin{proof}
Instead of counting numbers that are relatively prime to $pq$, we count all numbers that are
not relatively prime to $pq$. Since both $p$ and $q$ are prime numbers, the only numbers in
$[0, pq)$ that are not relatively prime to $pq$ is multiples of $p$ and multiples of $q$.
There are precisely $q$ multiples of $p$ and $p$ multiples of $q$ in $[0, pq)$. They both
count $0$ so we need to subtract $1$ from the result. So there are $p + q - 1$ numbers in
$[0, pq]$ that are not relatively prime to $pq$, that is
\begin{align*}
\Phi(pq) &= pq - (p + q - 1) \\
&=p(q-1) - (q - 1) \\
&=(p-1)(q-1)
\end{align*}
\end{proof}

\begin{theorem}
If $p$ is a prime, then $\Phi(p^k) = p^k - p^{k-1}$ for $k \geq 1$.
\end{theorem}

\begin{proof}
Since $p$ is a prime number, the only numbers that are not relatively prime to $p$ in
$[0, p^k)$ is multiples of $p$. That is, every $p$th number among the $p^k$ numbers is
divisible by $p$. So $1/p$ of these numbers are not relatively prime to $p$. That is,
\[
  \Phi(p^k) = p^k - (1/p)p^k = p^k - p^{k-1}
\]
\end{proof}

\begin{lemma} \label{lemma:33}
Suppose $a$ and $b$ are relatively prime and greater than $1$. The function $f:[0,ab)
\mapsto [0,a) \times [0, b)$ defined by
\[
  f(x) ::= (rem(x,a), rem(x,b))
\]
is a bijection.
\end{lemma}

\begin{proof}
By theorem \ref{theorem:2.8}, distinct $x \in [0, ab)$ maps to distinct $m \in [0,a)$ by
$rem(x,a)$ maps to distinct $n \in [0,b)$ by $rem(x,b)$.
\end{proof}

\begin{lemma} \label{lemma:34}
Suppose $a$ and $b$ are relatively prime and greater than $1$. The function $f:\Zs_{ab}
\mapsto \Zs_a \times \Zs_b$ defined by
\[
  f(x) ::=(rem(x,a),rem(x,b))
\]
is a bijection.
\end{lemma}

\begin{proof}
An obvious fact is that if $x$ is not relatively prime to $a$, then $x$ is not relatively
prime to $ab$. The contrapositive of it says that if $x$ is relatively prime to $ab$, then
$x$ is relatively prime to $x$. So take an element $x$ from $\Zs_{ab}$, $rem(x,a)$ is
relatively prime to $a$ and $rem(x,b)$ is relatively prime to $b$. So the expression given
in the lemma does a valid mapping.

And since $\Zs_{ab} \subseteq [0,ab)$, by lemma \ref{lemma:33}, the function defined in the
lemma is a bijection.
\end{proof}

\begin{theorem} \label{theorem:35}
Suppose $a$ and $b$ are relatively prime and greater than 1, then
$\Phi(ab) = \Phi(a)\Phi(b)$.
\end{theorem}

\begin{proof}
By lemma \ref{lemma:34}, there is a bijection between $\Zs_{ab}$ and $\Zs_a \times \Zs_b$. That is to say, $|\Zs_{ab}|=|\Zs_a \times \Zs_b|$, which implies that
$\Phi(ab)=\Phi(a)\Phi(b)$.
\end{proof}

\begin{corollary}
For any number $n > 1$, if $p_1,p_2,\dots,p_j$ are the distinct prime factors of $n$, then
\[
  \Phi(n)=n\left(1-\frac{1}{p_1}\right)\left(1-\frac{1}{p_2}\right)\dots\left(1-
  \frac{1}{p_j}\right)
\]
\end{corollary}

\begin{proof}
By theorem \ref{theorem:35} we have
\begin{align*}
\Phi(n) &= \Phi(p_1p_2 \dots p_j) \\
&= \Phi(p_1)\Phi(p_2)\dots\Phi(p_j) \\
&= (p_1-1)(p_2-1)\dots(p_j-1) \\
&= n\left(\frac{p_1-1}{p_1}\right)\left(\frac{p_2-1}{p_2}\right)\dots\left(
\frac{p_j-1}{p_j}\right)
\end{align*}

\end{proof}

\subsection{Bonus: The Chinese Remainder Theorem}

\begin{lemma} \label{lemma:2.8.1}
Suppose $a$ and $b$ are relatively prime and greater than $1$. For all $m,n$, there is an
$x$ such that
\begin{align*}
x &\equiv m \pmod{a} \\
x &\equiv n \pmod{b}
\end{align*}
\end{lemma}

\begin{proof}
Since $a$ and $b$ are relatively prime, $a$ has an inverse in $\mathbb{Z}_b$ and $b$ has an
inverse in $\mathbb{Z}_a$. We define $e_a ::= b^{-1}b$ and $e_b ::= a^{-1}a$. Let
$x = me_a + me_b$. We know that
\begin{align*}
rem(x,a) &= rem(me_a,a) + rem(me_b,a) \\
&=rem(m,a)rem(e_a,a) + rem(m,a)rem(e_b,a) \\
&=rem(m,a)rem(b^{-1}b,a) + rem(m,a)rem(a^{-1}a,a) \\
&=rem(m,a)
\end{align*}
That is to say $x \equiv m \pmod{a}$. And by the same reasoning we can show that
$x \equiv n \pmod{b}$.
\end{proof}

\begin{lemma} \label{lemma:2.8.2}
Suppose $a$ and $b$ are relatively prime and greater than $1$.
$x \equiv 0 \pmod{a} \wedge x \equiv 0 \pmod{b} \Rightarrow x \equiv 0 \pmod{ab}$.
\end{lemma}

\begin{proof}
Since $x \equiv 0 \pmod{a}$, $x = pa$. Since $x \equiv 0 \pmod{b}$, $x = qb$. Since $a$ and
$b$ are relatively prime, the only way $a|x$ and $a|qb$ is that $a|q$. Then we can write
$q=ra$ so $x=rab$. That means $ab|x$, which is also $x \equiv 0 \pmod{ab}$.
\end{proof}

\begin{lemma} \label{lemma:2.8.3}
Suppose $a$ and $b$ are relatively prime and greater than $1$.
$x \equiv x' \pmod{a} \wedge x \equiv x' \pmod{b} \Rightarrow x \equiv x' \pmod{ab}$.
\end{lemma}

\begin{proof}
Let $s = x - x'$. By lemma \ref{lemma:2.8.2},
\[
x-x' \equiv 0 \pmod{a} \wedge x-x' \equiv 0 \pmod{b} \Rightarrow x-x' \equiv 0 \pmod{ab}
\]

From the definition we know that $x-x' \equiv 0 \pmod{a} \Leftrightarrow p|(x-x')$. But
$p|(x-x') \Leftrightarrow x \equiv x' \pmod{a}$. So the implication above is just what want
to proof.
\end{proof}

\begin{theorem}[The Chinese Remainder Theorem] \label{theorem:2.8}
Suppose $a$ and $b$ are relatively prime and greater than $1$. For every $m,n$, there is an
$x$ such that
\begin{align*}
x &\equiv m \pmod{a} \\
x &\equiv n \pmod{b}
\end{align*}
Moreover, $x$ is unique to congruence modulo $ab$, namely, if $x'$ is also a solution, then
$x \equiv x' \pmod{ab}$.
\end{theorem}

\begin{proof}
By lemma \ref{lemma:2.8.1}, we know that there exists a solution $x$. Suppose that there is
another solution $x'$, where $x \equiv m \equiv x' \pmod{a}$ and $x \equiv n \equiv x'
\pmod{b}$. By lemma \ref{lemma:2.8.3} we know that $x \equiv x' \pmod{ab}$.
\end{proof}


\section{Graph Theory}

\subsection{Simple Graph}

\begin{definition}
A graph $G$ is a tuple $(V, E)$ where
\begin{itemize}
\item $V$ is a non-empty set of \emph{nodes} (or \emph{vertices})
\item $E \subseteq \{\{n,m\} \mid n,m \in V \text{ and } n \neq m\}$ is a set of edges
\end{itemize}
\end{definition}

\begin{definition}
Two nodes $n$ and $m$ are adjacent if there is an edge $\{n,m\} \in E$.
\end{definition}

\begin{definition}
An edge $e$ is \emph{incident} to a node $n$ if $n \in e$.
\end{definition}

\begin{definition}
The \emph{degree} of a node $n$ is $|V_n|$ where $V_n = \{e \in V \mid n \in e \}$.
\end{definition}

\begin{definition}
A graph is \emph{simple} if there are no loops and multi-edges.
\end{definition}

\begin{definition}[Graph Coloring Problem]
Given a graph $G$ and $k$ colors, assign a color to each node so that adjacent nodes get difffernt
colors. The minimum value of $k$ for which such a coloring exists is the \emph{chromatic number} of
$G$, denoted by $\chi(G)$.
\end{definition}

\begin{definition}
\emph{Basic Coloring Algorithm:}
\begin{enumerate}
\item order the nodes $v_1,v_2,\dots,v_n$
\item order the colors $c_1,c_2,\dots,c_m$
\item For $i=1,2,\dots,n$, assign the lowest legal color to $v_i$
\end{enumerate}
\end{definition}

\begin{theorem}
If every node of graph $G$ has degree $\leq d$, the Basic Coloring Algorithm uses at most $d+1$
colors.
\end{theorem}

\begin{proof}
We prove it by induction on $n$.

\textbf{Base Case:} For $n=1$, there are no edges. In this case $d = 0$ and $d+1=1$, which is
exactly the number of colors that we need.

\textbf{Inductive Case:} Assume $P(n)$ is true. Let $G=(V,E)$ be any $(n+1)$-node graph. Let $d$ be
the max degree of $G$. Order the nodes $v_1,v_2,\dots,v_n,v_{n+1}$. We remove $v_{n+1}$ from $G$ to
create $G'$. Then $G'$ has max degree $\leq d$. By the induction hypothesis, we can color it using
at most $d+1$ colors. Note that $v_{n+1}$ has at most $d$ neighbors and it takes at most $d$ colors
to color its neighbors. And we need one extra color for $v_{n+1}$, which implies we can color $G$
with $d+1$ colors.
\end{proof}

\begin{definition}
A \emph{bipartite graph} is a graph together with a partition of its vertices into two sets, $L$ and
$R$, such that every edge $L$ is incident to a vertex in $L$ and to a vertex in $R$.
\end{definition}

\subsection{Matching Problem}

\begin{definition}
Given a graph $G = (V, E)$, a \emph{matching} is a subgraph of $G$ where every node has degree $1$.
\end{definition}

\begin{definition}
A matching is \emph{perfect} if it has size $\frac{|V|}{2}$.
\end{definition}

\begin{definition}
The \emph{weight} of the matching $M$ is the sum of weights on the edges of $M$.
\end{definition}

\begin{definition}
A \emph{min-weight matching} for $G$ is a perfect matching for $G$ with the min weight.
\end{definition}

\begin{definition}
Given a matching $M$, $x,y$ form a \emph{rogue couple} if they prefer each other over their mates in
$M$.
\end{definition}

\begin{definition}
A matching is \emph{stable} if there are no rogue couples.
\end{definition}

\begin{definition}[Stable Marriage Problem]
Suppose there are $N$ boys and $N$ girls. Each boy has his own ranked preference list of all girls
and each girl has her own preference list of all boys. The goal is to find a perfect matching of
boys and girls.
\end{definition}

\begin{definition}[The Mating Ritual]
The procedure for finding a stable matching involves a \emph{Mating Ritual} that takes place over
several days. The following events happen each day:

\textbf{Morning:} Each boy serenades his favorite girl on the list. If a boy has no girls on his
list, he stays home and does his math homework.

\textbf{Afternoon:} Each girl who has one or more suitors says to her favorite among them, "we might
get engaged, come back tomorrow". To other suitors, she says, "I will never marry you".

\textbf{Evening:} Any boy who is told by a girl that she will never marry him, crosses that girl from
his list.

\textbf{Termination:} When a day arrives in which every girl has at most one suitor, the ritual ends
with every girl marrying her suitor, if she has one.
\end{definition}

\begin{theorem}
The Ritual eventually reaches the termination condition in less than $N^2+1$ days.
\end{theorem}

\begin{proof}
By contradiction. Suppose it does not terminate in $N^2+1$ days. That is to say there is a girl who
has more than one suitors. There are $N$ lists of $N$ names. But we have $N^2+1$ cross outs. So there
is a contradiction.
\end{proof}

\begin{lemma}
If a girl $G$ ever rejected a boy $B$, then $G$ has a suitor who she prefers to $B$.
\end{lemma}

\begin{proof}
By induction on the number of days.

\textbf{Base case:}  On day $0$, no one got rejected yet.

\textbf{Inductive case:} Assume $P$ holds at end of day $d$. There are two cases:
\begin{enumerate}
\item $G$ rejects $B$ at $d+1$. Then there was someone better. That implies $P$ is true for $d+1$.
\item $G$ rejects $B$ before $d+1$. Since $P$ is true at the end of day $d$, she has the same suitor
or someone she prefers better at the end of day $d+1$.
\end{enumerate}
\end{proof}

\begin{theorem}
Everyone is married.
\end{theorem}

\begin{proof}
By contradiction. Assume that some boy $B$ is not married. That is to say he is rejected by every
girl. And that means every girl has a better suitor and every girl is married. That is impossible
because the number of girls is the same as the number of boys.
\end{proof}

\begin{theorem}
The Ritual produces a stable matching.
\end{theorem}

\begin{proof}
By contradiction. Let $B$ and $G$ be any pair that are not married. There are several possible cases:
\begin{enumerate}
\item $G$ rejected $B$ because $G$ has a better suitor. $G$ marries someone better than $G$. $G$ and
$B$ are not rogue.
\item $G$ did not reject $B$. This means that $B$ never serenaded $G$ because $G$ is lower in $B$'s
list than $B$'s wife. So $G$ and $B$ are not rogue.
\end{enumerate}
\end{proof}

\begin{definition}
Let $S \neq \emptyset$ be the set of all stable matchings. For each person $P$, \emph{the realm of
possibility} for $P$ is $\{Q \mid \exists M \in S,\{P,Q\} \in M\}$.
\end{definition}

\begin{definition}
A person $P$'s \emph{optimal mate} is his/her favorite from the realm of possibility. Similarly we
can define a \emph{pessimal mate}.
\end{definition}

\begin{theorem} \label{theorem:64}
The Ritual marries every boy to his optimal mate.
\end{theorem}

\begin{theorem}
The Ritual marries every girl to her pessimal mate.
\end{theorem}

\begin{proof}
By contradiction. Suppose that there exists a stable matching $M$ where there is a girl $G$ who
marries to a boy $B$ that she likes less than her mate in the Ritual $B'$. This means that $G$
prefers $B'$ to $B$. By theorem \ref{theorem:64} and the fact that $G$ and $B'$ are married in the
Ritual, we know that $B'$ prefers $G$ to his mate in $M$. This means that $G$ and $B'$ are a rogue
couple in $M$, which contradicts the stability of $M$.
\end{proof}

\subsection{Spanning Trees}

\begin{definition}
A \emph{walk} is a sequence of vertices connected by edges. The \emph{length} of a walk is the number
of the edges.
\end{definition}

\begin{definition}
A \emph{path} is a walk in which all vertices are different from each other.
\end{definition}

\begin{lemma}
If there exists a walk from $u$ to $v$, then there also exists a path from $u$ to $v$.
\end{lemma}

\begin{proof}
By Well Ordering Principle. Suppose there is a walk of minimal length from $u$ to $v$. We are going
to prove this walk is a path. Suppose by contrary that the walk is not a path. Let us do a case
analysis on the length $k$ of the walk:
\begin{enumerate}
\item $k=0$. This is path.
\item $k=1$. The walk contains two different nodes and therefore it is a path.
\item $k \geq 2$. Suppose the walk is not a path. Then there is vertex occurred twice along the walk.
This means that there is a loop containing the vertex. Remove the loop and what's left is also a walk
from $u$ to $v$. This walk has a less length than the original one, which contradicts to our
assumption that the walk is a minimal one.
\end{enumerate}
\end{proof}

\begin{corollary}
The shortest walk between a pair of vertices is a path.
\end{corollary}

\begin{definition}
Two vertices $u$ and $v$ are \emph{connected} if there is path from $u$ to $v$.
\end{definition}

\begin{definition}
A graph is connected if every pair of vertices in the graph are connected.
\end{definition}

\begin{definition}
A \emph{closed walk} is a walk which starts and ends at the same vertex.
\end{definition}

\begin{definition}
A closed walk of length $k$ is said to be a \emph{cycle} if $k \geq 3$ and $v_0,v_1,\dots,v_{k-1}$
are all different.
\end{definition}

\begin{definition}
A \emph{tree} is a connected acyclic graph.
\end{definition}

\begin{definition}
A \emph{leaf} in a tree is a node with degree $1$.
\end{definition}

\begin{lemma}
Any connected subgraph of a tree is also a tree.
\end{lemma}

\begin{proof}
By contradiction. Suppose the connected subgraph is not a tree. That is to say there is a cycle in
it. If there is a cycle in the subgraph, the cycle must also be in the graph, which contradicts our
assumption that the graph is a tree.
\end{proof}

\begin{theorem}
If a tree has at least two vertices, it has at least two leaves.
\end{theorem}

\begin{proof}
Let $v_1,\dots,v_m$ be the sequence of vertices on a longest path in the tree. Then $m \geq 2$, since a
tree with two vertices must contain at least one edge. There cannot be an edge $\{v_1,v_i\}$ for
$2 < i \leq m$; otherwise vertices $v_1,\dots,v_i$ would form a cycle. Furthermore, there cannot be an
edge $\{u,v_1\}$ where $u$ is not on the path; otherwise, we could make the path longer. Therefore, the
only edge incident to $v_1$ is $\{v_1,v_2\}$, which means that $v_1$ is a leaf. By a symmetric argument,
$v_m$ is also a leaf.
\end{proof}

\begin{lemma}
A tree with $n$ vertices has $n-1$ edges.
\end{lemma}

\begin{proof}
By induction on $n$.

\textbf{Base case:} For $n=1$, there is no edges in a $1$-node tree.

\textbf{Inductive case:} Assume $P(n)$ holds. Let $T$ be a tree that has $n+1$ vertices. Let $v$ be a
leaf of the tree. Remove $v$ and we get a connected subgraph of $T$, which is a tree with $n$ nodes.
By the induction hypothesis this subtree has $n-1$ edges. Re-attach $v$ back to the tree, since it is
a leaf, we need one extra edge. So there are $n$ edges, which means that $P(n+1)$ is true.
\end{proof}

\begin{definition}
A \emph{spanning tree} (ST) of a connected graph is a subgraph that is a tree with the same vertices
as the graph.
\end{definition}

\begin{theorem}
Every connected graph has a spanning tree.
\end{theorem}

\begin{proof}
By contradiction. Suppose that there is a connected graph $G$ that has no spanning tree. Let $T$ be
a connected graph of $G$ with the same vertices as $G$ and with the smallest number of edges
possible. Since $T$ is not a spanning tree, there is a cycle. Let $e$ be an edge on the cycle and
$x,y$ are two vertices. There are two cases for $x$ and $y$:
\begin{enumerate}
\item The path from $x$ to $y$ does not contain $e$. In this case we can remove $e$ and $x$ and $y$
are still connected.
\item The path from $x$ to $y$ contains $e$. Since $e$ is an edge on the cycle, after removing $e$
there is still a path from $x$ to $y$ and they are connected.
\end{enumerate}
We have shown that all vertices in $G$ are still connected after remove $e$ from $T$, which
contradicts our assumption that $T$ is the smallest connected graph.
\end{proof}

\begin{definition}
A \emph{min-weight spanning tree (MST)} of an edge-weighted graph is the spanning tree of $G$ with
the smallest possible sum of edge weights.
\end{definition}

\begin{definition}[Algorithm to Find an MST]
Grow a subgraph one edge at a time by adding the minimum-weight edge possible to the subgraph,
making sure that we have an acyclic subgraph at each step.
\end{definition}

\begin{lemma} \label{lemma:83}
Let $S$ consists of the first $m$ edges selected by the algorithm. Then there exists some MST $T=
(V,E)$ for $G$ such that $S \subseteq E$.
\end{lemma}

\begin{proof}
By induction on $m$.

\textbf{Base case:} For $m=0$ it is correct.

\textbf{Inductive case:} Suppose $P(m)$ holds. Let $e$ denote the edge added in the $(m+1)$th step.
Let $S$ denote the first $m$ selected edges. Let $T'=(V,E')$ be the spanning tree guaranteed by the
induction hypothesis. There are two cases for $e$ and $E'$:
\begin{enumerate}
\item $e \in E'$. We are done.
\item $e \not\in E'$. We need to find a different MST that contains $S$ and $e$.
\end{enumerate}

Since $T'$ is a tree, adding an edge that is not in the tree would form a cycle. Moreover, the cycle
cannot contain only edges in $S$, since $e$ was chosen so that together with the edges in $S$ it does
not form a cycle. This implies the cycle contains an edge $e'$ of $E'-S$. Now we know that the weight
of $e'$ is no greater than $e$ and since $e' \in E'$ it cannot make a cycle with $S$.

Now we will make an MST that contains $S \cup \{e\}$. Let $T''=(V,E'')$ where $E''=(E'-\{e'\})\cup
\{e\}$. And we claim that $T''$ is an MST. Now we prove the claim.

First, we show that $T''$ is a spanning tree. $T''$ is acyclic because it is form by removing an edge
from the only cycle in $T' \cup \{e\}$. $T''$ is connected because the edge we deleted from $T'$ was on
a cycle. And $T''$ contains all the nodes of $G$. It is a spanning tree.

Then, we show that $T''$ is an MST. We build $T''$ by adding $e$ and removing $e'$. And by the
definition of the algorithm, $e$ is no greater than $e'$, so the weight of $T''$ is no greater than
$T'$, and thus $T''$ is an MST.

Now we have shown that $T''$ is an MST, and $S \cup \{e\} \subseteq E''$. So $P(m+1)$ holds. Our proof
by induction is done.
\end{proof}

\begin{theorem}
For any connected edge-weighted graph $G$, the algorithm gives an MST for $G$.
\end{theorem}

\begin{proof}
We need to show that the algorithm terminates and it creates an MST.

To show that the algorithm terminates, we need to show that it never gets stuck. By lemma
\ref{lemma:83}, as long as there are fewer than $n-1$ edges picked, there exists some edge in $E-S$ and
so there is an edge we can add to $S$ without forming a cycle.

To show that the algorithm creates an MST, following the last paragraph, once $m=n-1$, $S$ is a MST by
lemma \ref{lemma:83}.
\end{proof}

\subsection{Euler Tour and Directed Graph}

\begin{definition}
An \emph{Euler tour} is a walk that traverses every edges exactly once and starts and finishes at
the same vertex.
\end{definition}

\begin{theorem}
A connected graph has Euler tour iff every vertex of the graph has even degree.
\end{theorem}

\begin{proof}
First, let us prove the only if direction. Assume $G=(V,E)$ has an Euler tour. Suppose there is a
Euler tour starts and finishes at $v_0$. For any vertex $v_k$ in the Euler tour, it may occur
multiple times. Everytime it occurs, there are two edges incident to it. Since these edges are in
a Euler tour, they are all different and they are all edges that are incident to $v_k$. So the
degree of $v_k$ is two times the occurrences of $v_k$, which is an even number.

Then, let us prove the if direction. Assume for $G=(V,E)$, the degree of every vertex if even. Let
$w = v_0v_1 \dots v_k$ be the longest walk that traverses no edge more than once. We claim that
$w$ is an Euler tour.

Suppose that there is an edge between $v_k$ and $u$ that is not in the walk. We can attach the edge
to the walk and obtain a longer walk, which contradicts our assumption that $w$ is the longest
walk. So edges incident to $v_k$ are used in $w$.

Now we show that $v_k=v_0$. Suppose that $v_k \neq v_0$, then $v_k$ has an odd degree in $w$.
Since we have shown that all edges that are incident to $v_k$ are in $w$, $v_k$ must also has an
odd degree in $G$, which contradict our assumption that every vertex of the graph has even degree.
So $v_k$ must be $v_0$.

Suppose $w$ is not an Euler tour. $G$ is connected so there exists an edge that is not in $w$ but
incident to some vertex in $w$. Let $u$-$v_i$ be this edge. Then we can build a longer walk
$uv_iv_{i+1} \dots v_kv_0 \dots v_i$, which contradicts our assumption that $w$ is the longest walk
of $G$. So $w$ must be an Euler tour.
\end{proof}

\begin{theorem}
Let $G=(V,E)$ be an $n$-node graph. Let $A$ be the adjacent matrix of $G$. Let $p_{i,j}^k$ be the
number of directed walks from $v_i$ to $v_j$ of length $k$. Then $A^k = \{p_{i,j}^k\}$.
\end{theorem}

\begin{proof}
By induction on $k$. Assume $a_{i,j}^k$ be the $(i,j)$th element of $A^k$, we want to prove that
$a_{i,j}^k=p_{i,j}^k$.

\textbf{Base case:} $k=1$. The only walk of length $1$ from $v_i$ to $v_j$ is an edge. If there is
an edge, $p_{i,j}^k=1$. If there is no edge, $p_{i,j}^k=0$. In this case, $p_{i,j}^k$ is exactly
the definition of the adjacent matrix.

\textbf{Inductive case:} Assume $P(m)$. Note that a path from $i$ to $j$ of length $m+1$ can be
obtained by attaching one more node $h$ to a path from $i$ to $h$ of length $m$ where there exists
an edge from $h$ to $j$. And we sum up for all $h$s to get the final result:
\begin{align*}
p_{i,j}^{k+1} &= \sum_{h\text{ that there exists an edge from }h\text{ to }j} p_{ih}^k \\
&= \sum_{h=1}^n p_{ih}^k \cdot a_{hj} \\
&= \sum_{h=1}^n a_{ih}^k \cdot a_{hj}
\end{align*}
Note that the last summation is exactly the definition of matrix multiplication.

\end{proof}

\begin{definition}
A digraph $G=(V,E)$ is \emph{strongly connected} if for all $u,v \in V$ there exists a directed
path from $u$ to $v$ in $G$.
\end{definition}

\begin{definition}
A directed graph is called a \emph{directed acyclic graph (DAG)} if it does not contain any
directed cycles.
\end{definition}

\begin{definition}
A \emph{directed Hamiltonian path} is a directed path that visits every vertex exactly once.
\end{definition}

\begin{theorem}
Every tournament graph contains a directed Hamiltonian path.
\end{theorem}

\begin{proof}
By induction on the number of nodes.

\textbf{Base case:} $n=1$ works.

\textbf{Inductive case:} Assume $P(m)$. Take out one node $v$ and the remaining graph is an
$n$-node tournament graph. Let $w=v_0 \dots v_k$ be the directed Hamiltonian path guaranteed by the
induction hypothesis. There are two cases:
\begin{itemize}
\item $v$ beats $v_0$. We can insert $v$ before $v_0$ and obtain a directed Hamiltonian path.
\item $v_0$ beats $v$. Let $i$ be the smallest number that $v$ beats $v_i$. Since $v$ beats $v_i$
and $v_{i-1}$ beats $v$, we can insert $v$ between $v_{i-1}$ and $v_i$ and obtain a directed
Hamiltonian path.
\end{itemize}
\end{proof}

\subsection{Relations and Partial Orders}

\begin{definition}
A \emph{relation} from a set $A$ to a set $B$ is a subset $R \subseteq A \times B$.
\end{definition}

\begin{definition}
A relation on $A$ is
\begin{itemize}
\item \emph{reflexive} if $(x,x) \in R$ for all $x \in A$
\item \emph{symmetric} if $(x,y) \in R \Rightarrow (y,x) \in R$ for all $x,y \in A$
\item \emph{antisymmetric} if $(x,y) \in R \wedge (y,x) \in R \Rightarrow x = y$ for all
$x,y \in A$
\item \emph{transitive} if $(x,y) \in R \wedge (y,z) \in R \Rightarrow (x,z) \in R$ for all
$x,y,z \in A$
\end{itemize}
\end{definition}

\begin{definition}
An \emph{equivalence relation} is a relation that is reflexive, symmetric and transitive.
\end{definition}

\begin{definition}
The equivalence class of $x \in A$ is the set of all elements in $A$ that is equivalent to $x$ with
respect to $R$.
\end{definition}

\begin{definition}
A \emph{partition} of $A$ is a collection of disjoint nonempty sets $A_1,\dots,A_n \subseteq A$,
where $A_1\cup\dots\cup A_n = A$.
\end{definition}

\begin{theorem}
The equivalence classes of an equivalence relation on a set $A$ form a partition of $A$.
\end{theorem}

\begin{definition}
A \emph{partial order} is a relation that is reflexive, antisymmetric and transitive.
\end{definition}

\begin{definition}
A Hasse diagram for a poset $(A,\leq)$ is a digraph with vertices set $A$ and edge set $\leq$
without all self-loops and all edges implied by transitivity.
\end{definition}

\begin{theorem}
A poset has no directed cycles other than self-loops.
\end{theorem}

\begin{proof}
By contradiction. Suppose there is a cycle of $n$ distinct elements in $A$. Since $a_1 \leq a_n$
and $a_n \leq a_1$, $a_1$ must be equal to $a_n$, which contradicts our assumption that $a_1$ is
distinct to $a_n$.
\end{proof}

\begin{corollary}
A Hasse diagram is acyclic.
\end{corollary}

\begin{definition}
A \emph{total order} is a partial order in which every pair of elements is comparable.
\end{definition}

\begin{proposition}
The Hasse diagram of a total order is a straight line.
\end{proposition}

\begin{definition}
A \emph{topological sort} of a poset is a total order $(A,\leq_T)$ such that $\leq \in \leq_T$.
\end{definition}

\begin{definition}
$x \in A$ is called \emph{minimal} if there is no $y \neq x$ such that $y \leq x$.
\end{definition}

\begin{lemma}
Every finite poset has a minimal element.
\end{lemma}

\begin{theorem}
Every finite poset has a topological sort.
\end{theorem}

\end{document}
